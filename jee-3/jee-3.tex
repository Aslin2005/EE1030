\let\negmedspace\undefined
\let\negthickspace\undefined
\documentclass[journal,12pt,twocolumn]{IEEEtran}
\usepackage{cite}
\usepackage{amsmath,amssymb,amsfonts,amsthm}
\usepackage{algorithmic}
\usepackage{graphicx}
\usepackage{textcomp}
\usepackage{xcolor}
\usepackage{txfonts}
\usepackage{listings}
\usepackage{enumitem}
\usepackage{mathtools}
\usepackage{gensymb}
\usepackage{comment}
\usepackage[breaklinks=true]{hyperref}
\usepackage{tkz-euclide} 
\usepackage{listings}
\usepackage{gvv}
\def\inputGnumericTable{}
\usepackage[latin1]{inputenc}                                
\usepackage{color}                                            
\usepackage{array}                                            
\usepackage{longtable}                                       
\usepackage{calc}                                             
\usepackage{multirow}                                         
\usepackage{hhline}                                           
\usepackage{ifthen}                                           
\usepackage{lscape}

\newtheorem{theorem}{Theorem}[section]
\newtheorem{problem}{Problem}
\newtheorem{proposition}{Proposition}[section]
\newtheorem{lemma}{Lemma}[section]
\newtheorem{corollary}[theorem]{Corollary}
\newtheorem{example}{Example}[section]
\newtheorem{definition}[problem]{Definition}
\newcommand{\BEQA}{\begin{eqnarray}}
\newcommand{\EEQA}{\end{eqnarray}}
\newcommand{\define}{\stackrel{\triangle}{=}}
\theoremstyle{remark}
\newtheorem{rem}{Remark}
\usepackage{circuitikz}
\begin{document}

\bibliographystyle{IEEEtran}
\vspace{3cm}
\title{2023-April Session-04-12-2023-shift-1}
\author{EE24BTECH11008-ASLIN GARVASIS}
\maketitle
\begin{enumerate}[start=16]
    \item If $\frac{1}{n+1}^nC_n+\frac{1}{n}^nC_{n-1}+\cdot+\frac{1}{2}^nC_1+^nC_0=\frac{1023}{10}$ then $n$ is equal to $:$\\
    \begin{enumerate}
        \item $6$
        \item $9$
        \item $8$
        \item $7$
    \end{enumerate}
    \item Let $\textbf{C}$ be the circle in the complex plane with centre $z_0=\frac{1}{2}\brak{1+3i}$ and radius $r=1.$ Let $z_1=1+i$ and the complex number $z_2$ be outside the circle $\textbf{C}$ such that $\abs{z_1-z_0}\abs{z_2-z_0}=1.$ If $z_0,z_1$ and $z_2$ are collinear, then the smaller value of $\abs{z_2}^2$ is equal to $:$\\
    \begin{enumerate}
        \item $\frac{13}{2}$
        \item $\frac{5}{2}$
        \item $\frac{3}{2}$
        \item $\frac{7}{2}$
    \end{enumerate}
    \item If the point $\brak{\alpha,\frac{7\sqrt{3}}{3}}$ lies on the curve traced by the mid-points of the line segments of the lines $x\cos\theta+y\sin\theta=\gamma,$ $\theta\in \brak{0,\frac{\pi}{2}}$ between the co-ordinates axes, then $\alpha$ is equal to $:$\\
    \begin{enumerate}
        \item $7$
        \item $-7$
        \item $-7\sqrt{3}$
        \item $7\sqrt{3}$
    \end{enumerate}
    \item Two dice $A$ and $B$ are rolled. Let the numbers obtained on $A$ and $B$ be $\alpha$ and $\beta$ respectively. If the varience of $\alpha-\beta$ is $\frac{p}{q},$ where $p$ and $q$ are co-prime, then the sum of the positive divisors of $p$ is equal to $:$\\
    \begin{enumerate}
        \item $36$
        \item $48$
        \item $31$
        \item $72$
    \end{enumerate}
    \item In a triangle $ABC$ if $\cos A+2\cos B+\cos C=2$ and the lengths of the sides opposite to the angles $A$ and $C$ are $3$ and $7$ respectively, then $\cos A-\cos C$ is equal to $:$\\
    \begin{enumerate}
        \item $\frac{3}{7}$
        \item $\frac{9}{7}$
        \item $\frac{10}{7}$
        \item $\frac{5}{7}$
    \end{enumerate}
    \item A fair $n\brak{n \textgreater 1}$ faces die rolled repeatedly until a number less than $n$ appears. If the mean of the number of tosses required is $\frac{n}{9},$ then $n$ is equal to $\dots$\\
    \item Let the digits $a,b,c$ be in A.P. Nine-digit numbers are to be formed using each of three such that three consecutive digits are in A.P at least once. How many such numbers can be formed $?$\\
    \item Let $\sbrak{x}$ be the greatest integer $\le x.$ Then the number of points in the interval $\brak{-2,1},$ where the function $f\brak{x}=\abs{\sbrak{x}}+\sqrt{x-\sbrak{x}}$ is discontinuous is $\dots$\\
    \item Let the plane $x+3y-2z+6=0$ meet the coordinate axes at the points $A,B,C.$ If the orthocentre of the triangle $ABC$ is $\brak{\alpha,\beta,\frac{6}{7}},$ then $98\brak{\alpha+\beta}^2$ is equal to $\dots$\\
    \item Let $I\brak{x}=\int\sqrt{\frac{x+7}{x}}dx$ and $I\brak{9}=12+7\log_e7.$ If $I\brak{1}=\alpha+7\log_e\brak{1+2\sqrt{2}},$ then $\alpha ^4$ is equal to $\dots$\\
    \item Let $D_k=\myvec{1&2k&2k-1\\n&n^2+n+2&n^2\\n&n^2+n&n^2+n+2}.$ If $\sum_{k=1}^nD_k=96,$ then $n$ is equal to $\dots$\\
    \item Let the positive numbers $a_1,a_2,a_3,a_4$ and $a_5$ be in G.P. Let their mean and variance be $\frac{31}{10}$ and $\frac{m}{n}$ respectively,where $m$ and $n$ are co-prime. If the mean of their reciprocals is $\frac{31}{40}$ and $a_3+a_4+a_5=14,$ then $m+n$ is equal to $\dots$\\
    \item The number of relations, on the set ${1,2,3}$ containing $\brak{1,2}$ and $\brak{2,3},$ which are reflexive and transitive but not symmetic, is $\dots$\\
    \item If $\int_{-0.15}^{0.15}\abs{100x^2-1}dx=\frac{k}{3000},$ then $k$ is equal to $\dots$\\
    \item Two circles in the first quadrant of radii $r_1$ and $r_2$ touch the coordinate axes. Each of them cuts off an intercept of $2$ units with the line $x+y=2.$ Then $r_1^2+r_2^2-r_1r_2$ is equal to $\dots$
\end{enumerate}
\end{document}
