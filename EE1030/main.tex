\let\negmedspace\undefined
\let\negthickspace\undefined
\documentclass[journal,10pt,twocolumn]{IEEEtran}
\usepackage{cite}
\usepackage{amsmath,amssymb,amsfonts,amsthm}
\usepackage{algorithmic}
\usepackage{graphicx}
\usepackage{textcomp}
\usepackage{xcolor}
\usepackage{txfonts}
\usepackage{listings}
\usepackage{enumitem}
\usepackage{mathtools}
\usepackage{gensymb}
\usepackage{comment}
\usepackage[breaklinks=true]{hyperref}
\usepackage{tkz-euclide} 
\usepackage{listings}
\usepackage{gvv}
\def\inputGnumericTable{}
\usepackage[latin1]{inputenc}                                
\usepackage{color}                                            
\usepackage{array}                                            
\usepackage{longtable}                                       
\usepackage{calc}                                             
\usepackage{multirow}                                         
\usepackage{hhline}                                           
\usepackage{ifthen}                                           
\usepackage{lscape}

\newtheorem{theorem}{Theorem}[section]
\newtheorem{problem}{Problem}
\newtheorem{proposition}{Proposition}[section]
\newtheorem{lemma}{Lemma}[section]
\newtheorem{corollary}[theorem]{Corollary}
\newtheorem{example}{Example}[section]
\newtheorem{definition}[problem]{Definition}
\newcommand{\BEQA}{\begin{eqnarray}}
\newcommand{\EEQA}{\end{eqnarray}}
\newcommand{\define}{\stackrel{\triangle}{=}}
\theoremstyle{remark}
\newtheorem{rem}{Remark}
\usepackage{circuitikz}
\begin{document}

\bibliographystyle{IEEEtran}
\vspace{3cm}
\title{Chapter-$1$ Trignometric Functions and Equations}
\author{EE24BTECH11008-ASLIN GARVASIS}
\maketitle
\begin{enumerate}[start=14]
\item The expression $\frac{\tan A}{1-\cot A} +\frac{\cot A}{1-\tan A}$ can be written as:\\

\hfill {(JEE M 2013)}\\
    \begin{enumerate}
    \item $\sin\brak{A}\cos\brak{A}+1$
    \item $\sec\brak{A}\cosec\brak{A}+1$
    \item $\tan\brak{A}+\cot\brak{A}$ 
    \item $\sec\brak{A}+\cosec\brak{A}$
    \end{enumerate}
\item Let $f_{k}x=\frac{1}{k}$ $\brak{\sin^{k}x+\cos^{k}x}$ where $x\in R$ AND $k\geq 1 \cdot $\\
 Then $f_{4}\brak{x}-f_{6}\brak{x}$ equals

\hfill {(JEE M 2014)}\\
    \begin{enumerate}
    \item  $\frac{1}{4}$ 
     \item $\frac{1}{12}$
    \item $\frac{1}{6}$
    \item $\frac{1}{3}$
    \end{enumerate}
\item If $0 \ge x \ge 2\pi$, then the number of real values of x, which \\  satisfy the equation $\cos x+\cos2x+\cos3x+\cos4x=0$ is:

\hfill {(JEE M 2016)}\\
    \begin{enumerate}
    \item $7$
    \item $9$
    \item $3$
    \item $5$
    \end{enumerate}
    
\item If $5${$\tan^2x-\cos^2x=2\cos2x+9$} then value of $\cos 4x$ is:

\hfill{(JEE M 2017)}\\
    \begin{enumerate}
    \item $\frac{-7}{9}$ 
    \item $\frac{-3}{5}$
    \item $\frac{1}{3}$
    \item $\frac{2}{9}$\\
    \end{enumerate}
 \item If sum of all the solutions of the equation\\
  $8 \cos\brak{x} \cdot \cos\brak{\frac{\pi}{6} + x }$ $\cdot \cos \brak{\frac{\pi}{6}}$ - $\frac{1}{2} - 1 \text{ in }$ $\sbrak {0, \pi}$\text{ is } k $\pi$

 then k is equal to:\\
 
\hfill{(JEE M 2018)}\\
\begin{enumerate}
\item $\frac{13}{9}$
\item $\frac{8}{9}$\\
\item  $\frac{20}{9}$
\item  $\frac{2}{3}$\\
\end{enumerate}
  \item For any $\theta \in \brak{\frac{\pi}{4}}$,$\brak{\frac{\pi}{2}}$ the expression\\
 $3\brak{in\theta-\cos\theta^4 +6}$ $\brak{\sin\theta+\cos\theta^2 +4\sin^{6}\theta}$ equals:\\
 
\hfill {(JEE M 2019-9 Jan  M)}\\
 \begin{enumerate}
 \item $13-4\cos^2\theta +6\sin^2\theta \cos^2\theta $\\
 \item  $13-4\cos^6\theta$\\
\item  $13-4\cos^2\theta +6\cos^4\theta$\\
 \item $13-4\cos^2\theta +2\sin^2\theta \cos^2\theta$\\
 \end{enumerate}
\item The value of\\ $\cos^210^{0}-\cos10^{0}\cos50^{0}+\cos^250^{0}$ is:\\

\hfill {(JEE M 2019-9 April M)}\\
\begin{enumerate}
\item $\frac{3}{4}$ $+\cos20^0$
\item $\frac{3}{4}$\\
 \item $\frac{3}{2}$ $\brak{1+\cos20^0\quad}$ 
 \item $\frac{3}{2}$\\
 \end{enumerate}
\item Let S=$\theta \in \sbrak{-2\pi, 2\pi}$ :$2\cos^2\theta + 3\sin\theta=0.$\\
 Then the sum of the elements of S is:\\
 
\hfill {(JEE M 2019-9 April M)}\\
\begin{enumerate}
\item $\frac{13\pi}{6}$ 
\item $\frac{5\pi}{3}$
 \item $2$
 \item $1$\\
\end{enumerate} 
\end{enumerate}
\begin{center}
Chapter-4 Permutations and Combinations 
\end{center}
section-A JEE Advanced/IIT JEE   
\\
A Fill in the Blanks\\

\begin{enumerate}
\item In a certain test, $a_i$ students gave wrong answers to atleast 
 $i$ questions, where $i=1,2,...,k.$No student gave more than 
 $k$ wrong answers.The total number of wrong answers
 given is.......
 
\hfill {(1982-2 Marks)} \\

\item The side $AB,BC$ and $BC$ of a triangle $ABC$ have $3$,$4$ and $5$ 
 interior points respectively on them. The number of triangles
 that can be constructed using these interior points as
 vertices is......
 
\hfill {(1984-2 Marks)} \\

\item Total number of ways in which six $'+'$and four $'-'$signs can
 be arranged in a line such that no two $'-'$signs occur together
 is......
 
\hfill  {(1988-2 Marks)} \\

\item There are four balls of different colours and four boxes of
 colours, same as those of the balls. The number of ways in
 which the balls,one each in a box,could be placed such that
 a ball does not go to a box of its own colour is.......\\
\\
\end{enumerate} 

 B True/False \\
\begin{enumerate} 
\item The product of any $r$ consecutive natural numbers is always
 divisible by $r!.$
 
\hfill {(1985-1 Marks)}  \\
\end{enumerate} 
C MCQs with One Correct Answer\\
\\
\begin{enumerate}
\item  $^nC_{r-1}=36$,$^nC_r=84$ and $^nC_{r-1}=126$, then $r$ is:

\hfill {(1979-1 Marks)}\\
\begin{enumerate}
 \item $1$
 \item $2$
 \item $3$
\item  None of these
\end{enumerate}
\item Ten different letters of an alphabet are given. Words with
 five letters are formed from these given letters. Then the
 number of words which have at least one letter repeated are
 
\hfill {(1982-2 Marks)}\\
\begin{enumerate}
 \item $69760$
 \item $30240$
 \item $99748$
 \item none of the above
 \end{enumerate}
\end{enumerate}
\end{document}
