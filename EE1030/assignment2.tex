\let\negmedspace\undefined
\let\negthickspace\undefined
\documentclass[journal,10pt,twocolumn]{IEEEtran}
\usepackage{cite}
\usepackage{amsmath,amssymb,amsfonts,amsthm}
\usepackage{algorithmic}
\usepackage{graphicx}
\usepackage{textcomp}
\usepackage{xcolor}
\usepackage{txfonts}
\usepackage{listings}
\usepackage{enumitem}
\usepackage{mathtools}
\usepackage{gensymb}
\usepackage{comment}
\usepackage[breaklinks=true]{hyperref}
\usepackage{tkz-euclide} 
\usepackage{listings}
\usepackage{gvv}
\def\inputGnumericTable{}
\usepackage[latin1]{inputenc}                                
\usepackage{color}                                            
\usepackage{array}                                            
\usepackage{longtable}                                       
\usepackage{calc}                                             
\usepackage{multirow}                                         
\usepackage{hhline}                                           
\usepackage{ifthen}                                           
\usepackage{lscape}

\newtheorem{theorem}{Theorem}[section]
\newtheorem{problem}{Problem}
\newtheorem{proposition}{Proposition}[section]
\newtheorem{lemma}{Lemma}[section]
\newtheorem{corollary}[theorem]{Corollary}
\newtheorem{example}{Example}[section]
\newtheorem{definition}[problem]{Definition}
\newcommand{\BEQA}{\begin{eqnarray}}
\newcommand{\EEQA}{\end{eqnarray}}
\newcommand{\define}{\stackrel{\triangle}{=}}
\theoremstyle{remark}
\newtheorem{rem}{Remark}
\usepackage{circuitikz}
\begin{document}

\bibliographystyle{IEEEtran}
\vspace{3cm}
\title{Chapter-$16$ Application of Derivatives}
\author{EE24BTECH11008-ASLIN GARVASIS}
\maketitle
\begin{enumerate}[start=9]
\item The slope of the tangent to a curve $y=f\brak{x}$ at $\sbrak{x,f\brak{x}}$ is $2x+1.$If the curve passes through the point $\brak{1,2},$ then the area bounded by the curve, the $x$ axis and the line $x=1$ is

\hfill {(1995S)} \\

\begin{enumerate}
    \item $\frac{5}{6}$
    \item $\frac{6}{5}$
    \item $\frac{1}{6}$
    \item $6$
\end{enumerate}
\item If $f\brak{x}=\frac{x}{\sin x}$ and $g\brak{x}=\frac{x}{\tan x}$, where$0$ \textless $x \le 1$, then in this interval

\hfill {(1997-2 Marks)} \\

\begin{enumerate}
    \item both $f\brak{x}$ and $g\brak{x}$ are increasing functions
    \item both $f\brak{x}$ and $g\brak{x}$ are decreasing functions 
    \item $f\brak{x}$ is an increasing function
    \item $g\brak{x}$ is an increasing function 
\end{enumerate}
\item The function $f\brak{x}=\sin^4x+\cos^4x$ increases if

\hfill {(1999-2 Marks)} \\

\begin{enumerate}
    \item $0$ \textless $x$ \textless $\frac{\pi}{8}$ 
    \item $\frac{\pi}{4}$ \textless $x$ \textless $\frac{3\pi}{8}$
    \item $\frac{3\pi}{8}$ \textless $x$ \textless $\frac{5\pi}{8}$
    \item $\frac{5\pi}{8}$ \textless $x$ \textless $\frac{3\pi}{4}$
\end{enumerate}
\item Consider the following statements in S and R \hfill{(2000S)}\\
\subsubsection*{S} Both $\sin x$ and $\cos x$ are decreasing functions in the interval $\brak{\frac{\pi}{2},\pi}$
\subsubsection*{R} If a differentiable function decreases in an interval $\brak{a,b}$, then its derivative also decreases in $\brak{a,b}$\\
Which of the following is true $?$

\begin{enumerate}
    \item Both S and R are wrong
    \item Both S and R are correct, but R is not the correct explanation of S
    \item S is correct and R is the correct explanation for S
    \item S is correct and R is wrong
\end{enumerate}
\item Let $f\brak{x}=\int{e^x\brak{x-1}\brak{x-2}dx}$ Then f decreases in the interval
\hfill {(2000S)} \\
\begin{enumerate}
    \item $\brak{-\infty,-2}$
    \item $\brak{-2,-1}$
    \item $\brak{1,2}$
    \item $\brak{2,+\infty}$
\end{enumerate}
\item If the normal to the curve $y=f\brak{x}$ at the point $\brak{3,4}$ makes an angle $\frac{3\pi}{4}$ with the positive $x-$ axis, then $f'\brak{3}=$
\hfill {(2000S)} \\
\begin{enumerate}
    \item $-1$
    \item $\frac{-3}{4}$
    \item $\frac{4}{3}$
    \item $1$
\end{enumerate}
\item $$ Let f(x) = 
\begin{cases} 
\abs{x} & \text{for } 0 \textless x \leq 2, \\
1 & \text{for } x = 0 
\end{cases}
$$
Then at $x=0, f$ has 

\hfill {(2000S)} \\

\begin{enumerate}
    \item a local maximum
    \item a local minimum
    \item no local maximum
    \item no extremum
\end{enumerate}
\item For all $x \in \brak{0,1}$ \hfill {(2000S)} \\
\begin{enumerate}
    \item $e^x$\textless$1+x$
    \item $\log_e\brak{1+x}$\textless $x$
    \item $\sin x$\textgreater $x$
    \item $\log_e\brak{x}$\textgreater $x$
\end{enumerate}
\item If $f\brak{x}=xe^{x\brak{1-x}}$, then $f\brak{x}$ is \hfill {(2001S)} \\
\begin{enumerate}
    \item increasing on $\sbrak{\frac{-1}{2},1}$
    \item decreasing on R
    \item increasing on R
    \item decreasing on $\sbrak{\frac{-1}{2},1}$
\end{enumerate}
\item The tiangle formed by the tangent to the curve $f\brak{x}=x^2+bx-b$ at the point \brak{1,1}and the coordinate axes, lies in the first quadrant.If its area is $2$, then the value of$b$ is \hfill {(2001S)} \\
\begin{enumerate}
    \item $-1$
    \item $3$
    \item $-3$
    \item $1$
\end{enumerate}
\item Let $f\brak{x}=\brak{1+b^2}x^2+2bx+1$ and let $m\brak{b}$ be the minimum value of $f\brak{x}.$As $b$ varies, the range of $m\brak{b}$ is \hfill {(2001S)} \\
\begin{enumerate}
    \item $\sbrak{0,1}$
    \item $\brak{0,\frac{1}{2}}$
    \item $\sbrak{\frac{1}{2},1}$
    \item $\brak{0,1}$
\end{enumerate}
\item The length of a longest interval in which the function $3\sin x-4\sin^3x$ is increasing, is \hfill {(2002S)} \\
\begin{enumerate}
    \item $\frac{\pi}{3}$
    \item $\frac{\pi}{2}$
    \item $\frac{3\pi}{2}$
    \item $\pi$
\end{enumerate}
\item The point$\brak{s}$ on the curve $y^3+3x^2=12y$ where the tangent is vertical, is$\brak{are}$ \hfill{(2002S)} \\
\begin{enumerate}
    \item $\brak{\pm \frac{4}{\sqrt{3}},-2}$
    \item $\brak{\pm \sqrt{\frac{11}{3}},1}$
    \item $\brak{0,0}$
    \item $\brak{\pm \frac{4}{\sqrt{3}},2}$
\end{enumerate}
\item In $\sbrak{0,1}$ Lagranges Mean Value Theorem is NOT applicable to \hfill {(2003S)} \\
\begin{enumerate}
    \item $$  f(x) = 
\begin{cases} 
\frac{1}{2}-x &  x \textless \frac{1}{2}\\
\brak{\frac{1}{2}-x}^2 & x \geq \frac{1}{2}
\end{cases}
$$
    \item $$  f(x) = 
\begin{cases} 
\frac{\sin x}{x} &  x \neq 0\\
1 & x = 0
\end{cases}
$$
    \item $f\brak{x}=x\abs{x}$
    \item $f\brak{x}=\abs{x}$
\end{enumerate}
\item A tangent is drawn to ellipse $\frac{x^2}{27}+y^2=1$ at $\brak{3\sqrt{3}\cos \theta, \sin \theta}$ $\brak{where \theta \in \brak{0,\frac{\pi}{2}}.}$ Then the value of $\theta$ such that sum of intercepts on axes made by this tangent is minimum, is \hfill {(2003S)} \\
\begin{enumerate}
    \item $\frac{\pi}{3}$
    \item $\frac{\pi}{6}$
    \item $\frac{\pi}{8}$
    \item $\frac{\pi}{4}$
\end{enumerate}
\end{enumerate}
\end{document}
