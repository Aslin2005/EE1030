\let\negmedspace\undefined
\let\negthickspace\undefined
\documentclass[journal,15pt,onecolumn]{IEEEtran}
\usepackage{cite}
\usepackage{amsmath,amssymb,amsfonts,amsthm}
\usepackage{algorithmic}
\usepackage{graphicx}
\usepackage{textcomp}
\usepackage{xcolor}
\usepackage{txfonts}
\usepackage{listings}
\usepackage{enumitem}
\usepackage{mathtools}
\usepackage{gensymb}
\usepackage{comment}
\usepackage[breaklinks=true]{hyperref}
\usepackage{tkz-euclide} 
\usepackage{listings}
\usepackage{gvv}
\def\inputGnumericTable{}
\usepackage[latin1]{inputenc}                                
\usepackage{color}                                            
\usepackage{array}                                            
\usepackage{longtable}                                       
\usepackage{calc}                                             
\usepackage{multirow}                                         
\usepackage{hhline}                                           
\usepackage{ifthen}                                           
\usepackage{lscape}

\newtheorem{theorem}{Theorem}[section]
\newtheorem{problem}{Problem}
\newtheorem{proposition}{Proposition}[section]
\newtheorem{lemma}{Lemma}[section]
\newtheorem{corollary}[theorem]{Corollary}
\newtheorem{example}{Example}[section]
\newtheorem{definition}[problem]{Definition}
\newcommand{\BEQA}{\begin{eqnarray}}
\newcommand{\EEQA}{\end{eqnarray}}
\newcommand{\define}{\stackrel{\triangle}{=}}
\theoremstyle{remark}
\newtheorem{rem}{Remark}
\usepackage{circuitikz}

\begin{document}

\bibliographystyle{IEEEtran}
\vspace{3cm}
\title{2007-EE}
\author{EE24BTECH11008-ASLIN GARVASIS}
\maketitle
\begin{enumerate}[start=69]
    \item Which one of the following statements regarding the INT $\brak{\text{interrupt}}$ and the BRQ $\brak{\text{bus request}}$ pins in a CPU is true $?$\\
    \begin{enumerate}
        \item The BRQ pin is sampled after every instruction cycle, but the INT is sampled after every machine cycle
        \item Both INT and BRQ are sampled after every machine cycle
        \item The INT pin is sampled after every instruction cycle, but the BRQ is sampled after every machine cycle
        \item Both INT and BRQ are sampled after every instruction cycle
    \end{enumerate}
    \item A bridge is shown in the figure below. Which one of the sequences given below is most suitable for balancing the bridge $?$
\begin{circuitikz}
    % Define the nodes and components
    \draw
    (2,2) to[R=$R_1$] (3,3)  % Resistor R1 from bottom left to top left
    (0.5,0.5) to[L=$jX_1$] (1.5,1.5)
    (0,0) -- (0.5,0.5)
    (1.5,1.5) -- (2,2)
    (3,3) -- (3.5,3.5)
    (3.5,3.5) -- (5,2)
    (6,1) -- (7,0)
    (0,0) -- (1,-1)
    (2,-2) -- (3.5,-3.5)
    (3.5,-3.5) -- (4,-3)
    (5,-2) -- (5.5,-1.5)
    (6.5,-0.5) -- (7,0)
    (0,0) -- (-1,0)
    (7,0) -- (8,0)
    (-1,0) -- (-1,-4)
    (8,0) -- (8,-4)
    (-1,-4) -- (3,-4)
    (8,-4) -- (5,-4)
    (1,-1) to[R=$R_3$] (2,-2) % Resistor R3 from bottom left to bottom right
    (5,2) to[R=$R_2$] (6,1)  % Resistor R2 from top left to right
    (5.5,-1.5) to[R=$R_4$] (6.5,-0.5) % Resistor R4 from bottom right to right
    (4,-3) to[C=$-jX_4$] (5,-2)
           % Straight line from right to ground
    (3.5,3.5) -- (3.5,2) % Vertical line connecting R1 and R3 (Bridge)
    (3.5,-3.5) -- (3.5,-2)
    % Voltage source on the left side of the bridge
    (3,-4) to[sV, l=$V_{in}$] (5,-4); % Sinusoidal voltage source
\end{circuitikz}
\begin{enumerate}
    \item First adjust $R_4$ and then adjust $R_1$
    \item First adjust $R_2$ and then adjust $R_3$
    \item First adjust $R_2$ and then adjust $R_4$
    \item First adjust $R_4$ and then adjust $R_2$
\end{enumerate}
\end{enumerate}
\begin{center}
    \textbf{Common Data Questions}
\end{center}
\textbf{Common Data for Questions $71,72,73 $ $:$}\\
A three phase squirrel cage induction motor has a starting current of seven times the full load current and full load slip of $5\%$ 
\begin{enumerate}[start=71]
    \item If an autotransformer is used for reduced voltage starting to provide $1.5$ per unit starting torque, the transformer ratio $\brak{\%}$ should be
    
    \begin{enumerate}
        \item $57.77\%$
        \item $72.56\%$
        \item $78.25\%$
        \item $81.33\%$
    \end{enumerate}
    \item If a star-delta starter is used to start this induction motor, the per unit starting torque will be
    \begin{enumerate}
        \item $0.607$
        \item $0.816$
        \item $1.225$
        \item $1.616$
    \end{enumerate}
    \item If a starting torque of $0.5$ per unit is required then the per unit starting current should be 
    \begin{enumerate}
        \item $4.65$
        \item $3.75$
        \item $3.16$
        \item $2.13$
    \end{enumerate}
\end{enumerate}
\textbf{Coomon Data for Questions $74,75$ $:$}\\
A $!:1$ Pulse Transformer $\brak{\text{PT}}$ is used to trigger the SCR in the adjacent figure. The SCR is rated at $1.5\brak{kV},$ $250\brak{A}$ with $I_L=250\brak{mA},I_H=150\brak{mA},$ and $I_{Gmax}=150,I_{Gmin}=100\brak{mA}.$ The SCR is connected to an inductive load, where L=$150\brak{mH}$ in series with a small resistance and the supply voltage is $200\brak{V}$ dc. The forward drops of all transistors/diodes and gate-cathode junction during ON state are $1.0\brak{0V}.$
\begin{figure}[!ht]
\centering
\resizebox{0.3\textwidth}{!}{%
\begin{circuitikz}
\tikzstyle{every node}=[font=\small]
\draw (7,14.25) to[L ] (7,12);
\draw (7.5,12) to[short] (7.5,14.25);
\draw (7.75,14.25) to[short] (7.75,12);
\draw (8.25,14.25) to[L ] (8.25,12);
\node [font=\normalsize] at (7.5,14.75) {PT};
\draw (7,14.25) to[short] (6,14.25);
\draw (6,13.5) to[D] (6,14.25);
\draw (6,13.5) to[R] (6,12.5);
\draw (6,12.5) to[short] (6,12);
\draw (6,12) to[short] (7,12);
\draw (8.25,14.25) to[D] (9,14.25);
\draw (9,14.25) to[short] (9,13.25);
\draw (9,12.75) to[D] (9,13.25);
\draw (9,12.75) to[short] (9,12);
\draw (8.25,12) to[short] (9,12);
\draw (9,14.25) to[R] (10,14.25);
\draw (10,14.25) to[short] (10.5,14.25);
\draw (10.5,14.25) to[short] (10.5,13.5);
\draw (10.5,13.5) to[C] (10.5,12.75);
\draw (9,12) to[short] (10.5,12);
\draw (10.5,12.75) to[short] (10.5,12);
\draw (10.5,12) to[short] (11,12);
\draw (11,14.75) to[short] (11,12);
\draw (11,12) to[short] (11,11.5);
\draw (10.5,14.25) to[short] (11,14.75);
\draw (11,15.25) to[D] (11,14.5);
\draw (11,15.25) to[short] (11,15.75);
\draw (11,15.75) to[short] (11.5,15.75);
\draw (11.5,15.75) to[R,l={ \small 1.0$\Omega$}] (12.5,15.75);
\draw (12.5,15.75) to[short] (13,15.75);
\draw (13,15.75) to[short] (13,14.25);
\draw (13,14.25) to[L,l={ \small L} ] (13,13);
\draw (13,13) to[battery1,l=$+200 V$] (13,11.5);
\draw (11,11.5) to[short] (13,11.5);
\draw (6.5,12) to[short] (6.5,11.5);
\draw (6,11.75) to[short] (6,10.25);
\draw (5.25,11) to[short] (6,11);
\draw (6,11) to[short] (6.5,11.5);
\draw (6.5,10.75) to[short] (6.5,10);
\draw [->, >=Stealth] (6,11) -- (6.5,10.75);
\draw [ line width=1.5pt](6.25,10) to[short] (6.75,10);
\draw [line width=0.6pt, ->, >=Stealth] (6.5,14.25) -- (6.5,14.75);
\node [font=\small] at (6,15) {+10 V};

\node [font=\small] at (9.5,14.75) {R};
\node [font=\small] at (11.5,15.25) {\textbf{S}};
\node [font=\small] at (11.5,15) {\textbf{C}};
\node [font=\small] at (11.5,14.75) {\textbf{R}};
\end{circuitikz}
}%

\label{fig:my_label}
\end{figure}
\begin{enumerate}[start=74]
    \item The resistor R should be
    \begin{enumerate}
        \item $4.4\brak{k\ohm}$
        \item $470\brak{\ohm}$
        \item $47\brak{\ohm}$
        \item $4.7\brak{\ohm}$
    \end{enumerate}
    \item The minimum approximate volt-second rating of the pulse transformer suitable for triggering the SCR should be$;$ $\brak{\text{volt-second rating is the maximum of product of the voltage and the width of the pulse that may be applied}}$
    \begin{enumerate}
        \item $2000\brak{\mu V-s}$
        \item $200\brak{\mu V-s}$
        \item $20\brak{\mu V-s}$
        \item $2.0\brak{\mu V-s}$
    \end{enumerate}
\end{enumerate}
\begin{center}
    \textbf{Linked Answer Questions: Q$.76$ to Q$.85$ carry two marks each.}
\end{center}
\textbf{Statement for Linked Answer Questions $76 \& 77:$}\\
An inductor designed with $400$ turns coil wound on an iron core of $16\brak{cm^2}$ cross sectional area with a cut of an air gap length of $1\brak{mm}.$ The coil is connected to a $230\brak{V},5050\brak{Hz}$ ac supply. Neglect coil resistence, core loss, iron reluctance and leakage inductance. $\brak{\mu_0=4\pi\times10^{-7}H/m}$
\begin{enumerate}[start=76]
    \item The current in the inductor is
    \begin{enumerate}
        \item $18.08\brak{A}$
        \item $9.04\brak{A}$
        \item $4.56\brak{A}$
        \item $2.28\brak{A}$
    \end{enumerate}
    \item The average force on the core to reduce the air gap will be
    \begin{enumerate}
        \item $832.29\brak{N}$
        \item $1666.22\brak{N}$
        \item $3332.47\brak{N}$
        \item $6664.84\brak{N}$
    \end{enumerate}
\end{enumerate}
\textbf{Statement for Linked Answer Qestions $78 \& 79;$}\\
Cayley-Hamilton Theorem states that a square matrix satisfies its own characteristic equation.\\
Consider a matrix
\begin{center}
    $$A=\myvec{-3&2\\-1&0}$$
\end{center}
\begin{enumerate}[start=78]
    \item A satisfies the relation
    \begin{enumerate}
        \item $A+3I+2A^{-1}=0$
        \item $A+2I+2A=0$
        \item $\brak{A+I}\brak{A+2I}=0$
        \item $\exp\brak{A}$
    \end{enumerate}
    \item $A^9$ equals
    \begin{enumerate}
        \item $511A+310I$
        \item $309A+104I$
        \item $154A+155I$
        \item $\exp\brak{9A}$
    \end{enumerate}
\end{enumerate}
\textbf{Statement for Linked Answer Questions $80 \& 81$:}\\
Consider the R-L-C circuit shown in figure
\begin{figure}[!ht]
\centering
\resizebox{0.3\textwidth}{!}{%
\begin{circuitikz}
\tikzstyle{every node}=[font=\normalsize]
\draw [ line width=0.3pt](6.5,14.5) to[short] (6.5,13.25);
\draw [ line width=0.3pt ] (6.5,13) circle (0.25cm);
\draw [ line width=0.3pt](6.5,12.75) to[short] (6.5,12);
\draw [ line width=0.3pt](6.5,14.5) to[R,l={ \small R=10$\Omega$}] (8.5,14.5);
\draw [line width=0.3pt](8.5,14.5) to[L,l={ \small L=1mH} ] (10.25,14.5);
\draw [line width=0.3pt](10.25,14.5) to[C,l={ \small C=10$\mu$ F}] (10.25,13);
\draw [ line width=0.3pt](10.25,13) to[short] (10.25,12);
\draw [ line width=0.3pt](6.5,12) to[short] (10.25,12);
\node [font=\normalsize] at (5.75,13) {$e_i$};
\node [font=\normalsize] at (12,13.25) {$e_0$};
\draw [line width=0.3pt, ->, >=Stealth] (12,13.5) -- (12,14.5);
\draw [line width=0.3pt, ->, >=Stealth] (12,13) -- (12,12);
\end{circuitikz}
}%

\label{fig:my_label}
\end{figure}
\begin{enumerate}[start=80]
    \item For a step-input $e_i$ the overshoot in the output $e_0$ will be
    \begin{enumerate}
        \item $0,$ since the system is not under-damped
        \item $5\%$
        \item $16\%$
        \item $48\%$
    \end{enumerate}
    \item If the above step responce is to be observed on a non-storage CRO, then it would be best to have the $e_i$ as a
    \begin{enumerate}
        \item step function
        \item square wave of frequency $50\brak{Hz}$
        \item square wave of frequency $300\brak{Hz}$
        \item square wave of frequency $2.0\brak{kHz}$
    \end{enumerate}
\end{enumerate}
\textbf{Statement for Linked Answer Questions $82 \& 83:$}\\
The associated figure shows the two types of rotate right instructions $R1,$ $R2$ available in a microscope where Reg is a $8-$ bit register and C is the carry bit. The rotate left instructions $L1$ and $L2$ are similar except that C now links the most significant Reg instead of the least significant one.
\begin{figure}[!ht]
\centering
\resizebox{0.3\textwidth}{!}{%
\begin{circuitikz}
\tikzstyle{every node}=[font=\normalsize]



\draw [ line width=0.3pt ] (7,15) rectangle (10.25,14.5);
\draw [ line width=0.3pt ] (7,13.75) rectangle (10.25,13.25);
\draw [ line width=0.3pt ] (11.5,15) rectangle (12,14.5);
\draw [ line width=0.3pt ] (11.5,13.75) rectangle (12,13.25);
\node [font=\normalsize] at (8.5,14.75) {$Reg$};
\node [font=\normalsize] at (8.5,13.5) {$Reg$};
\node [font=\normalsize] at (11.75,14.75) {$C$};
\node [font=\normalsize] at (11.75,13.5) {$C$};
\draw [ line width=0.3pt](12,14.75) to[short] (12.5,14.75);
\draw [ line width=0.3pt](12.5,14.75) to[short] (12.5,14.25);
\draw [ line width=0.3pt](12.5,14.25) to[short] (6.5,14.25);
\draw [ line width=0.3pt](6.5,14.75) to[short] (6.5,14.25);
\draw [ line width=0.3pt](10.25,13.5) to[short] (10.75,13.5);
\draw [ line width=0.3pt](10.75,13.5) to[short] (10.75,13);
\draw [ line width=0.3pt](10.75,13) to[short] (6.5,13);
\draw [ line width=0.3pt](6.5,13) to[short] (6.5,13.5);
\draw [line width=0.3pt, ->, >=Stealth] (6.5,14.75) -- (7,14.75);
\draw [line width=0.3pt, ->, >=Stealth] (10.25,14.75) -- (11.5,14.75);
\draw [line width=0.3pt, ->, >=Stealth] (10.75,13.5) -- (11.5,13.5);
\draw [line width=0.3pt, ->, >=Stealth] (6.5,13.5) -- (7,13.5);
\end{circuitikz}
}%

\label{fig:my_label}
\end{figure}
\begin{enumerate}[start=82]
    \item Suppose Reg contains the $2\prime s$ complement number $111010110.$ If this number is divided by $2$ the answer should be
    \begin{enumerate}
        \item $01101011$
        \item $10010101$
        \item $11101001$
        \item $11101011$
    \end{enumerate}
    \item Such a division can be correctly performed by the following set of operations
    \begin{enumerate}
        \item $L2,R2,R1$
        \item $L2,R1,R2$
        \item $R2,L1,R2$
        \item $R1,L2,R2$
    \end{enumerate}
\end{enumerate}
\textbf{Statement Linked Answer Questions $84 \& 85:$}\\
\begin{enumerate}[start=84]
    \item A signal is processed by a casual filter with transfer function $G\brak{s}.$ For a distortion free output signal waveform, $G\brak{s}$ must
    \begin{enumerate}
        \item provide zero phase shift for all frequency
        \item provide constant phase shift for all frequency
        \item provide linear phase shift that is proportional to frequency
        \item provide a phase shift that is inversely proportional to frequency
    \end{enumerate}
    \item $G\brak{z}=\alpha z^{-1}+\beta z^{-3}$ is a low-pass digital filter with a phase characteristics same as that of the above question if
    \begin{enumerate}
        \item $\alpha =\beta$
        \item $\alpha =-\beta$
        \item $\alpha =\beta^{\brak{\frac{1}{3}}}$
        \item $\alpha =\beta^{\brak{\frac{-1}{3}}}$
    \end{enumerate}
\end{enumerate}
\end{document}
