\let\negmedspace\undefined
\let\negthickspace\undefined
\documentclass[journal,12pt,twocolumn]{IEEEtran}
\usepackage{cite}
\usepackage{amsmath,amssymb,amsfonts,amsthm}
\usepackage{algorithmic}
\usepackage{graphicx}
\usepackage{textcomp}
\usepackage{xcolor}
\usepackage{txfonts}
\usepackage{listings}
\usepackage{enumitem}
\usepackage{mathtools}
\usepackage{gensymb}
\usepackage{comment}
\usepackage[breaklinks=true]{hyperref}
\usepackage{tkz-euclide} 
\usepackage{listings}
\usepackage{gvv}
\def\inputGnumericTable{}
\usepackage[latin1]{inputenc}                                
\usepackage{color}                                            
\usepackage{array}                                            
\usepackage{longtable}                                       
\usepackage{calc}                                             
\usepackage{multirow}                                         
\usepackage{hhline}                                           
\usepackage{ifthen}                                           
\usepackage{lscape}

\newtheorem{theorem}{Theorem}[section]
\newtheorem{problem}{Problem}
\newtheorem{proposition}{Proposition}[section]
\newtheorem{lemma}{Lemma}[section]
\newtheorem{corollary}[theorem]{Corollary}
\newtheorem{example}{Example}[section]
\newtheorem{definition}[problem]{Definition}
\newcommand{\BEQA}{\begin{eqnarray}}
\newcommand{\EEQA}{\end{eqnarray}}
\newcommand{\define}{\stackrel{\triangle}{=}}
\theoremstyle{remark}
\newtheorem{rem}{Remark}
\usepackage{circuitikz}
\begin{document}

\bibliographystyle{IEEEtran}
\vspace{3cm}
\title{2021-March Session-03-16-2021-shift-1-16-30}
\author{EE24BTECH11008-ASLIN GARVASIS}
\maketitle
\begin{enumerate}[start=16]
    \item Let $\sbrak{x}$ denote greatest integer less than or equal to $x.$ If for $n \in \textbf{N},$ $\brak{1-x+x^3}^n=$ $\sum_{j=0}^{3n}a_jx^j$, then $\sum_{j=0}^{\sbrak{\frac{3n}{2}}}a_{2j}$$+4$$\sum_{j=0}^{\sbrak{\frac{3n-1}{2}}}a_{2j}+1$ is equal to $:$\\
    \begin{enumerate}
        \item $2$
        \item $2^{n-1}$
        \item $1$
        \item $n$
    \end{enumerate}
    \item If $y=y\brak{x}$ is the solution of the differential equation, $\frac{dy}{dx}+2y\tan x=\sin x,y\brak{\frac{\pi}{3}}=0,$ then the maximum value of the function $y\brak{x}$ over $R$ is equal to $:$\\
    \begin{enumerate}
        \item $8$
        \item $\frac{1}{2}$
        \item $-\frac{15}{4}$
        \item $\frac{1}{8}$
    \end{enumerate}
    \item The locus of the midpoints of the chord of the circle, $x^2+y^2=25$ which is tangent to the hyperbola, $\frac{x^2}{9}-\frac{y^2}{16}=1$ is $:$\\
    \begin{enumerate}
        \item $\brak{x^2+y^2}^2-16x^2+9y^2=0$
        \item $\brak{x^2+y^2}^2-9x^2+144y^2=0$
        \item $\brak{x^2+y^2}^2-9x^2-16y^2=0$
        \item $\brak{x^2+y^2}^2-9x^2+16y^2=0$
    \end{enumerate}
    \item The number of roots of the equation, $\brak{81}^{\sin^2x} +\brak{81}^{\cos^2x} =30$ in the interval $\sbrak{0, \pi}$ is equal to $:$\\

    \begin{enumerate}
        \item $3$
        \item $4$
        \item $8$
        \item $2$
    \end{enumerate}
    \item Let $\textbf{S}_k=\sum_{r=1}^k\tan^{-1}\brak{\frac{6^r}{2^{2r+1}+3^{2r+1}}}.$ Then $\lim_{k \to \infty}\textbf{S}_k$ is equal to $:$\\
    \begin{enumerate}
        \item $\tan^{-1}\brak{\frac{3}{2}}$
        \item $\frac{\pi}{2}$
        \item $\cot^{-1}\brak{\frac{3}{2}}$
        \item $\tan^{-1}\brak{3}$
    \end{enumerate}
    \item Consider an arithmetic series and a geometric series having four initial terms from the set $\sbrak{11,8,21,16,26,32,4}.$ If the last terms of these series are the maximum possible four digit numbers, then the number of common terms in these two series is equal to $\dots$\\
    \item Let $f :\brak{0,2} \to \textbf{R}$ be defined as $$f\brak{x}=\log_2\brak{1+\tan\brak{\frac{\pi x}{4}}}.$$ Then, $\lim_{n \to \infty}\frac{2}{n}\brak{f\brak{\frac{1}{n}}+f\brak{\frac{2}{n}}+ \dots +f\brak{1}}$ is equal to $\dots$\\
    \item Let $ABCD$ be a square of side of unit length. Let a circle $C_{1}$ centered at $A$ with unit radius is drawn. Another circle $C_{2}$ which touches $C_{1}$ and the lines $AD$ and $AB$ are tangent to it, is also drawn. Let a tangent line from the point $C$ to the circle $C_{2}$ meet the side $AB$ at $E$. If the length of $EB$ is $\alpha + \sqrt{3}\beta ,$ where $\alpha,\beta$ are integers, then $\alpha +\beta$ is equal to $\dots$\\

    \item If $\lim_{x \to 0}\frac{ae^x-b\cos x+ce^{-x}}{x\sin x}=2,$ then $a+b+c$ is equal to $\dots$\\
    \item The total number of $3 X 3$ matrices $A$ having entries from the set \brak{0,1,2,3} such that the sum of all the diagonal entries of $AA^T$ is $9,$ is equal to $\dots$\\
    \item Let $$\text{P} = \begin{bmatrix}
-30 & 20  & 56 \\
90  & 140 & 112 \\
120 & 60  & 14
\end{bmatrix} \text{ and A} =
\begin{bmatrix}
2  & 7  & \omega^2 \\
-1 & -\omega & 1 \\
0  & -\omega & -\omega + 1
\end{bmatrix}$$ where $\omega =\frac{-1+i\sqrt{3}}{2},$ and $\textbf{I}_3$ be the identity matrix of order $3.$ If the determinant of the matrix $\brak{P^{-1}AP-\textbf{I}_3}^2$ is $\alpha\omega^2,$ then the value of $\alpha$ is equal to $\dots$\\
\item If the normal to the curve $y\brak{x}=\int_0^x\brak{2t^2-15t+10t}dt$ at a point $\brak{a,b}$ is parallel to the line $x+3y=-5, a>1,$ then the value of $\abs{a+6b}$ is equal to $\dots$\\
\item Let the curve $y=y\brak{x}$ be the solution of the differential equation, $\frac{dy}{dx}=2\brak{x+1}.$ If the numerical value of area bounded by the curve $y=y\brak{x}$ and $x-$ axis is $\frac{4\sqrt{8}}{3},$ then the value of $y\brak{1}$ is equal to $\dots$\\
\item Let $f : \textbf{R} \to \textbf{R}$ be a continous function such that $f\brak{x}+f\brak{x+1}=2,$ for all $x \in \textbf{R}.$ If $\textbf{I}_1=\int_0^xf\brak{x}dx$ and $\textbf{I}_2=\int_{-1}^3f\brak{x}dx,$ then the value of $\textbf{I}_1 +2\textbf{I}_2$ is equal to $\dots$\\
\item Let $z$ and $w$ be two complex numbers such that $w=z\overline{z}-2z+2, \abs{\frac{z+i}{z-3i}}=1$ and $Re\brak{w}$ has the minimum value. Then the minimum value of $n \in \textbf{N}$ for which $w^n$ is real, is equal to $\dots$
\end{enumerate}
\end{document}
