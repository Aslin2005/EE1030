\let\negmedspace\undefined
\let\negthickspace\undefined
\documentclass[journal]{IEEEtran}
\usepackage[a5paper, margin=10mm, onecolumn]{geometry}
%\usepackage{lmodern} % Ensure lmodern is loaded for pdflatex
\usepackage{tfrupee} % Include tfrupee package

\setlength{\headheight}{1cm} % Set the height of the header box
\setlength{\headsep}{0mm}     % Set the distance between the header box and the top of the text

\usepackage{gvv-book}
\usepackage{gvv}
\usepackage{cite}
\usepackage{amsmath,amssymb,amsfonts,amsthm}
\usepackage{algorithmic}
\usepackage{graphicx}
\usepackage{textcomp}
\usepackage{xcolor}
\usepackage{txfonts}
\usepackage{listings}
\usepackage{enumitem}
\usepackage{mathtools}
\usepackage{gensymb}
\usepackage{comment}
\usepackage[breaklinks=true]{hyperref}
\usepackage{tkz-euclide} 
\usepackage{listings}
% \usepackage{gvv}                                        
\def\inputGnumericTable{}                                 
\usepackage[latin1]{inputenc}                                
\usepackage{color}                                            
\usepackage{array}                                            
\usepackage{longtable}                                       
\usepackage{calc}                                             
\usepackage{multirow}                                         
\usepackage{hhline}                                           
\usepackage{ifthen}                                           
\usepackage{lscape}
\begin{document}

\bibliographystyle{IEEEtran}
\vspace{3cm}
\title{1.1.5.7}
\author{EE24BTECH11008 - Aslin Garvasis
}
% \maketitle
% \newpage
% \bigskip
{\let\newpage\relax\maketitle}

\renewcommand{\thefigure}{\theenumi}
\renewcommand{\thetable}{\theenumi}
\setlength{\intextsep}{10pt} % Space between text and floats


\numberwithin{equation}{enumi}
\numberwithin{figure}{enumi}
\renewcommand{\thetable}{\theenumi}
 \textbf{Question:}If $\vec{A}\brak{\frac{a}{3},4}$ is the midpoint of the line segment joining the points $\vec{B}\brak{-6,5}$ and $\vec{C}\brak{-2,3},$  
		then the value of $a$ is\\
 
\textbf{solution:}\\
\begin{table}[h!]    
  \centering
  \begin{tabular}[12pt]{ |c| c|c|}
    \hline
        \textbf{Variable}  & \textbf{Description} & \textbf{formula}\\
    \hline
        $\vec{B}$$\brak{-6,5}$ &  coordinates of first point & $-$ \\
    \hline 
        $\vec{C}$$\brak{-2,3}$ & coordinates of second point & $-$\\
    \hline
        $\vec{A}$& midpoint of $\vec{B}$ and $\vec{C}$ & $-$\\ 
    \hline
        $k$ & ratio in which $\vec{c}$ divides the line joining $AB$ & $\frac{\vec{B}+k\vec{A}}{k+1}$\\
    \hline       
\end{tabular}

  \caption{Variables Used}
  \label{tab1.1.5.7}
\end{table}
\begin{align}
	k&=1\\
 \vec{A}&=\frac{\vec{B}+\vec{C}}{2}\\ 
\implies \vec{A}&=\frac{\myvec{-6\\5}+\myvec{-2\\3}}{2}=\frac{\myvec{-8\\8}}{2}=\myvec{-4\\4}\\
\because \vec{A}&=\myvec{\frac{a}{3}\\4}\\
\implies a&=-4\times3\\ 
 a&=-12
\end{align}
		\newpage


		\begin{figure}[h!]
                \centering
               \includegraphics[width=0.7\linewidth]{figs/Fig1.png}
			\caption{Plot of points $\vec{A}$, $\vec{B}$ and $\vec{C}$}
               
               \end{figure}
\end{document}
